%
% LO, Li-yu 
%

\documentclass[a4paper,12pt]{article}

\usepackage{graphicx}
\usepackage[english]{babel}
\usepackage[]{amsmath}
\usepackage{amsfonts}
\usepackage{mathtools}
\usepackage{amssymb}
\usepackage[a4paper, portrait, margin=1.0in]{geometry}
\usepackage{hyperref}
\usepackage{algorithm,algpseudocode}
\usepackage{indentfirst}
\usepackage{soul}
\usepackage{color}


%
\begin{document}
%
   \title{\textbf{Safety-Aware Autonomous Racing Takeover Via Sampled-Based MPC and Control Barrier Functions}  \\
   ME555 Automotive Engineering Design \\
   Proposal}
      
   \author{
      Patrick Li-Yu Lo \& Miles Xingyu Ye \\ 
    %   22039044R/20997405 \\ 
    %   e-mail: patty.lo@connect.polyu.hk lloac@connect.hkust.hk
      }          
   \date{}

   \maketitle

   \section{Abstract}
    In the course of autonomous racing, 

   \section{Research Objective}
   \begin{itemize}
    \item To develop an autonomous vehicle takeover system for autonomous racing scenarios, with potential application to the Indy Autonomous Challenge, in support of Duke Motorsport. 
    \item To implement Model Predictive Path Integral (MPPI) control \cite{williams2015model}, a sample-based model predictive control framework, to achieve optimal takeover maneuvers.
    \item To integrate Control Barrier Function (CBF) \cite{ames2019control} constraints within the control architecture to ensure safety guarantees during vehicle operation.
    \item To incorporate motion prediction based on temporal learning methods to inform MPPI-based trajectory generation.
    \item To establish simulation environments for validation of the proposed methodology and to facilitate future research and development.
    \item To integrate the simulation framework with Simulink and conduct experimental validation on two F1TENTH vehicles, subject to time availability.
   \end{itemize}


   \section{Related Work}


   

%    \section*{Project Brief Description}
%    Further detail will be elucidated in the final report.

\bibliographystyle{IEEEtran}
\bibliography{references}

\end{document}

